\section{Fundamentação teórica}
\subsection{Democracia e o Sistema eleitoral democrático}
Diversas formas de governo se estabeleceram ao longo dos séculos, 
as primeiras que se tem indicio foram elaboradas por Aristóteles discípulo
de Platão , posteriormente outros pensadores como Maquiavel e Montesquieu se propuseram
a pensar as formas de governo, e até certo ponto chegaram em pensamentos conclusivos semelhantes.
Montesquieu menciona a democracia como uma das principais possíveis formas de governo. 
Para Montesquieu, “quando, numa república, o povo como um todo possui o poder soberano,
trata-se de uma democracia” \cite{de1857oeuvres}. Com isso, podemos perceber 
claramente que para que haja uma sociedade democrática, a população deve exercer
o poder sobre o Estado. \par
Como bem se sabe a palavra democracia é oriunda do Grego \emph{(demos, povo; kratos, poder)} 
que significa \emph{poder do povo}. Baseado no conceito fundamental da democracia
pode-se estar governando uma só pessoa ou um grupo de representantes, desde que
a escolha destes individuos seja realmente do povo é considerado uma forma de governo
democratico \cite[A democracia]{ribeiro2001democracia}. \par 
A principal referencia que temos em democracia com certeza foi o regime de Atenas, 
o próprio povo ou a própria nação, composta dos cidadãos que a formavam,
decidia em praça pública seus problemas fundamentais, com base nisto é razoavel
imaginar que não era um modelo escalavel, ou seja conforme a população se tornava 
mais numerosa, mais complicada tornavam-se as reuniões em praças públicas. 
Calcula-se que o número de cidadãos, que decidiam os assuntos do Estado,
não ia muito adiante de 40000, embora nas assembléias não comparecessem mais de
alguns milhares de cidadãos. Não se poderia pensar em democracia direta com os
Estados modernos, seu território e a população de cada um dêles. 
Daí a necessidade de um regime, que dispensasse a presença da totalidade dos 
cidadãos, sem eliminar o direito de pronunciamento de cada um dêles. 
Para chegar a êsse resultado é que se foi imposto o Regime Representativo democratico,
em que o cidadão elege os governantes e êstes decidem em seu nome ou em seu lugar.
Foi uma fórmula prática para a solução do problema. quando se entendeu que era
preciso convocar o povo, para que o govêrno se exercesse em seu nome e, 
de alguma forma, com a sua responsabilidade ou a sua intervenção.
\cite[Eleição e Sistemas Eleitorais]{sobrinho1958eleiccao} \par
A democracia está quase sempre presente nas falas dos principais
líderes dos mais diversos países, sua importância na consagração da legitimidade
dos governantes, faz com que mesmo regimes não democraticos recorram a processos
eleitorais, buscando trazer para o regime apoio popular.\par
\begin{quote}
"Parodiando La Rochefoucauld, poderíamos dizer que este uso das eleições, nos 
regimes ditatoriais, constitui uma homenagem que os governos não democráticos
rendem aos princípios cardeais da democracia". \cite[Eleição e Sistemas Eleitorais]{sobrinho1958eleiccao}.
\end{quote} \par
Levando-nôs a conclusão de que apenas uma eleição não garante a existência de um
regime democrático. É preciso também que a eleição seja realizada dentro de um 
rigoroso processo com determinadas garantias que assegurem a liberdadade do voto
e a autenticidade das apurações; daí, também, a atenção que essas técnicas têm
merecido, no estudo e aplicação das leis, que organizam e disciplinam os pleitos. \par

\clearpage